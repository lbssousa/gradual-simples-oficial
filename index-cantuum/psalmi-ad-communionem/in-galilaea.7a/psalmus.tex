\SetVerseAnnotation{Sl 16(15),1b.2--4a.5--11}

\SetLinkTextL{\Antiphona{In Galilǽa}}
\SetLinkTextR{\Antiphona{Na Galileia}}

\SetVersePairs{
  % 3
  {\Inchoatio{In}[ ]{san}ctos, qui sunt in terra, \MediatioVII{ín}{clitos }{vi}[ros,] omnis volúntas \TerminatioVII{me}{a in }{e}os.}%
    {\Inchoatio{Aos}[ ]{san}tos, que estão na terra, pes\MediatioVII{so}{as hon}{ra}[das,] vai toda a \TerminatioVII{mi}{nha es}{ti}ma.},
  % 4a
  {\Inchoatio{Mul}{ti}plicántur do\MediatioVII{ló}{res e}{ó}[rum,] qui post deos aliénos acce\TerminatioVII{le}{ra}{vé}runt.}%
    {\Inchoatio{Mul}{ti}plicam-se as \MediatioVII{do}{res da}{que}[les] que correm a\-trás de \TerminatioVII{ou}{tros }{deu}ses.},
  % 5
  {\Inchoatio{Dó}{mi}nus pars hereditátis meæ et \MediatioVII{cá}{licis }{me}[i:] tu es qui détines \TerminatioVII{sor}{tem }{me}am.}%
    {\Inchoatio{Se}{nhor}, porção de minha herança e \MediatioVII{mi}{nha }{ta}[ça,] vós tendes na mão a parte \TerminatioVII{que}{ me }{ca}be.},
  % 6
  {\Inchoatio{Fu}{nes} cecidérunt mihi in præcláris; ínsuper et heréditas mea speciósa est mihi.}%
    {\Inchoatio{Pa}{ra} mim as cordas caíram em terrenos \MediatioVII{a}{pra}{zí}[veis;] sim, minha herança é preciosa \TerminatioVII{pa}{ra }{mim}.},
  % 7
  {\Inchoatio{Be}{ne}dícam Dóminum, qui tríbuit mihi \MediatioVII{in}{tel}{léc}[tum;] ínsuper et in nóctibus erudiérunt me \TerminatioVII{re}{nes }{me}i.}%
    {\Inchoatio{Ben}{di}go ao Senhor, que me deu in\MediatioVII{te}{li}{gên}[cia;] sim, até de noite ensinam-me \TerminatioVII{os}{ meus }{rins}.},
  % 8
  {\Inchoatio{Pro}{po}nébam Dóminum in conspéctu \MediatioVII{me}{o }{sem}[per;] quóniam a dextris est mihi, non \TerminatioVII{com}{mo}{vé}bor.}%
    {\Inchoatio{Eu}[ ]{sem}pre tinha o Senhor \MediatioVII{an}{te meus }{o}[lhos;] porque ele está à minha direita, não serei \TerminatioVII{a}{ba}{la}do.},
  % 9
  {\Inchoatio{Prop}{ter} hoc lætátum est cor \Flexa{me}[um,] et exsultavérunt præ\MediatioVII{cór}{dia }{me}[a;] ínsuper et caro mea requi\TerminatioVII{és}{cet }{in} spe.}%
    {\Inchoatio{Por}[ ]{is}so, alegrou-se meu cora\Flexa{ção}[,] minhas entranhas \MediatioVII{e}{xul}{ta}[ram,] e minha carne repousa na \TerminatioVII{es}{pe}{\-ran}\-ça.},
  % 10
  {\Inchoatio{Quó}{ni}am non derelínques ánimam meam \MediatioVII{in}{ in}{fér}[\-no] nec dabis sanctum tuum vidére cor\TerminatioVII{rup}{ti}{ó}\-nem.}%
    {\Inchoatio{Pois}[ ]{não} abandonareis minha alma \MediatioVII{nos}{ in}{fer}[nos,] e não permitireis que vosso santo veja a \TerminatioVII{cor}{rup}{ção}.},
  % 11
  {\Inchoatio{No}{tas} mihi fácies vias \Flexa{vi}[tæ,] plenitúdinem lætítiæ cum \MediatioVII{vul}{tu }{tu}[o,] delectatiónes in déxtera tua \TerminatioVII{us}{que in }{fi}nem.}%
    {\Inchoatio{Vós}[ ]{me} fareis conhecer os caminhos da \Flexa{vi}[da,] a plenitude da alegria com a \MediatioVII{vos}{sa }{fa}[ce,] as delícias à vossa direita, \TerminatioVII{pa}{ra }{sem}pre.}
}