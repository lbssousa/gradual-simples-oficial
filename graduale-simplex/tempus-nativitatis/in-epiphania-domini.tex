% !TeX root = ../fasciculo-3/a4.tex
% chktex-file 1

\subsection{Entrada}\label{subsection:tempus-nativitatis/in-epiphania-domini/introitus}
\MakeChantAntiphonPsalm{venite-adoremus-eum.8G}{psalmi-ad-introitum}

\AllowPageFlush

\subsection[Salmo Responsorial]{Salmo Responsorial \textmd{E 5}}\label{subsection:tempus-nativitatis/in-epiphania-domini/psalmus-responsorius}
\MakeChantPsalmTwoVerses{psalmi-responsorii}{adorabunt-eum.E5}

\subsection{Aleluia}\label{subsection:tempus-nativitatis/in-epiphania-domini/alleluia}
\MakeChantAlleluiaPsalm{3g}{afferte-domino.3g}

\AllowPageFlush

\subsection[Salmo Aleluiático]{Salmo Aleluiático \textmd{C 4}}\label{subsection:tempus-nativitatis/in-epiphania-domini/psalmus-alleluiaticus}
\begin{center}
  \begin{rubrica}
    O primeiro {\normalfont\Rbar} pode ser cantado apenas pelo grupo de cantores ou por todos. O segundo {\normalfont\Rbar} é cantado por todos.
  \end{rubrica}
\end{center}
\MakeChantPsalmOneVerse{psalmi-alleluiatici}{afferte-domino.C4}

\AllowPageFlush

\subsection{Ofertório}\label{subsection:tempus-nativitatis/in-epiphania-domini/offertorium}
\MakeChantAntiphonPsalm{reges-tharsis.1a2}{psalmi-ad-offertorium}

\AllowPageBreak

\subsection{Comunhão}\label{subsection:tempus-nativitatis/in-epiphania-domini/communio}
\MakeChantAntiphonPsalm{vidimus-stellam-eius.4E}{psalmi-ad-communionem}

\AllowPageFlush