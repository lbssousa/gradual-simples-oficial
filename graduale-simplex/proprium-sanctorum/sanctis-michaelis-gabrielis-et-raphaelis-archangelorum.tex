% !TeX root = ../../a4.tex
% chktex-file 1

\subsection{Entrada}\label{subsection:proprium-sanctorum/sanctis-michaelis-gabrielis-et-raphaelis-archangelorum/introitus}
\MakeChantAntiphonPsalm{benedicite-dominum.8G}{psalmi-ad-introitum}

\AllowPageFlush

\subsection[Salmo Responsorial]{Salmo Responsorial \textmd{E 5 \protect\GreStar}}\label{subsection:proprium-sanctorum/sanctis-michaelis-gabrielis-et-raphaelis-archangelorum/psalmus-responsorius}
\MakeChantPsalmTwoVerses{psalmi-responsorii}{laudate-dominum-de-caelis.E5}

\subsection{Aleluia}\label{subsection:proprium-sanctorum/sanctis-michaelis-gabrielis-et-raphaelis-archangelorum/alleluia}
\MakeChantAlleluiaPsalm{3g}{benedic-anima-mea-domino.3g}

\AllowPageFlush

\subsection[Salmo Aleluiático]{Salmo Aleluiático \textmd{C 4}}\label{subsection:proprium-sanctorum/sanctis-michaelis-gabrielis-et-raphaelis-archangelorum/psalmus-alleluiaticus}
\begin{rubrica}
    O primeiro {\normalfont\Rbar} pode ser cantado apenas pelo grupo de cantores ou por todos. O segundo {\normalfont\Rbar} é cantado por todos.
\end{rubrica}
\MakeChantPsalmOneVerse{psalmi-alleluiatici}{benedic-anima-mea-domino.C4}

\AllowPageFlush

\subsection{Ofertório}\label{subsection:proprium-sanctorum/sanctis-michaelis-gabrielis-et-raphaelis-archangelorum/offertorium}
\MakeChantAntiphonPsalm{stetit-angelus.4A}{psalmi-ad-offertorium}

\AllowPageFlush

\subsection{Comunhão}\label{subsection:proprium-sanctorum/sanctis-michaelis-gabrielis-et-raphaelis-archangelorum/communio}
\MakeChantAntiphonPsalm{omnes-angeli-eius.5a}{psalmi-ad-communionem}