% !TeX root = ../../a4.tex
% chktex-file 1

\subsection{Entrada}\label{subsection:communia/commune-bmv/introitus}
\MakeChantAntiphonPsalm{beata-mater.2D}{psalmi-ad-introitum}

\nobreaksubsection{Salmo Responsorial}
\begin{rubrica}
  Ver Solenidade da Imaculada Conceição de Nossa Senhora, página~\pageref{subsection:proprium-sanctorum/in-conceptione-immaculata-bmv/psalmus-responsorius}.
\end{rubrica}

\nobreaksubsection{Aleluia}
\begin{rubrica}
  Ver Solenidade da Imaculada Conceição de Nossa Senhora, página~\pageref{subsection:proprium-sanctorum/in-conceptione-immaculata-bmv/alleluia}.
\end{rubrica}

\nobreaksubsection{Salmo Aleluiático}
\begin{rubrica}
  Ver Solenidade da Imaculada Conceição de Nossa Senhora, página~\pageref{subsection:proprium-sanctorum/in-conceptione-immaculata-bmv/psalmus-alleluiaticus}.
\end{rubrica}

\nobreaksubsection{Ofertório}
\begin{rubrica}
  Ver Solenidade da Anunciação do Senhor, página~\pageref{subsection:proprium-sanctorum/in-annuntiatione-domini/offertorium}.
\end{rubrica}

\nobreaksubsection{Comunhão I}
\begin{rubrica}
  Ver Festa da Natividade de Nossa Senhora, página~\pageref{subsection:proprium-sanctorum/in-nativitate-bmv/communio}.
\end{rubrica}

\AllowPageFlush

\subsection{Comunhão II}\label{subsection:communia/commune-bmv/communio-2}
\MakeChantAntiphonPsalm{sanctum-nomen-domini.1g}{psalmi-ad-communionem}