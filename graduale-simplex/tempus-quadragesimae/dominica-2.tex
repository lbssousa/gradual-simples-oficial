% !TeX root = ../../a4.tex
% chktex-file 1

\subsection{Entrada}\label{subsection:proprium-sanctorum/in-transfiguratione-domini/introitus}
\MakeChantAntiphonPsalm{assumpsit-iesus.2D}{psalmi-ad-introitum}

\AllowPageFlush

\subsection[Salmo Responsorial I]{Salmo Responsorial I \textmd{C 2 a}}\label{subsection:proprium-sanctorum/in-transfiguratione-domini/psalmus-responsorius}
\MakeChantPsalmTwoVerses{psalmi-responsorii}{magna-est-gloria-eius.C2a}

\AllowPageFlush

\nobreaksubsection{Salmo Responsorial II}
\begin{rubrica}
  Ver III Domingo, página~\pageref{subsection:tempus-quadragesimae/dominica-1/psalmus-responsorius-2}
\end{rubrica}

\nobreaksubsection{Antífona de aclamação}
\begin{rubrica}
  Ver III Domingo, página~\pageref{subsection:tempus-quadragesimae/dominica-1/antiphona-acclamationis}
\end{rubrica}

%\nobreaksubsection{Trato}
%\begin{rubrica}
%  Ver III Domingo, página~\pageref{subsection:tempus-quadragesimae/dominica-1/tractus}
%\end{rubrica}

\subsection{Ofertório}\label{subsection:proprium-sanctorum/in-transfiguratione-domini/offertorium}
\MakeChantAntiphonPsalm{faciamus-hic.4A}{psalmi-ad-offertorium}

\AllowPageFlush

\subsection{Comunhão}\label{subsection:proprium-sanctorum/in-transfiguratione-domini/communio}
\MakeChantAntiphonPsalm{visionem.1f}{psalmi-ad-communionem}